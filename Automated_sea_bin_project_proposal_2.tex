%%%%%%%%%%%%  Generated using docx2latex.com  %%%%%%%%%%%%%%

%%%%%%%%%%%%  v2.0.0-beta  %%%%%%%%%%%%%%

\documentclass[12pt]{article}
\usepackage{amsmath}
\usepackage{latexsym}
\usepackage{amsfonts}
\usepackage[normalem]{ulem}
\usepackage{soul}
\usepackage{array}
\usepackage{amssymb}
\usepackage{extarrows}
\usepackage{graphicx}
\usepackage[backend=biber,
style=numeric,
sorting=none,
isbn=false,
doi=false,
url=false,
]{biblatex}\addbibresource{bibliography.bib}

\usepackage{subfig}
\usepackage{wrapfig}
\usepackage{wasysym}
\usepackage{enumitem}
\usepackage{adjustbox}
\usepackage{ragged2e}
\usepackage[svgnames,table]{xcolor}
\usepackage{tikz}
\usepackage{longtable}
\usepackage{changepage}
\usepackage{setspace}
\usepackage{hhline}
\usepackage{multicol}
\usepackage{tabto}
\usepackage{float}
\usepackage{multirow}
\usepackage{makecell}
\usepackage{fancyhdr}
\usepackage[toc,page]{appendix}
\usepackage[hidelinks]{hyperref}
\usetikzlibrary{shapes.symbols,shapes.geometric,shadows,arrows.meta}
\tikzset{>={Latex[width=1.5mm,length=2mm]}}
\usepackage{flowchart}\usepackage[paperheight=11.69in,paperwidth=8.27in,left=1.0in,right=1.0in,top=1.0in,bottom=1.0in,headheight=1in]{geometry}
\usepackage[utf8]{inputenc}
\usepackage[T1]{fontenc}
\TabPositions{0.5in,1.0in,1.5in,2.0in,2.5in,3.0in,3.5in,4.0in,4.5in,5.0in,5.5in,6.0in,}

\urlstyle{same}


 %%%%%%%%%%%%  Set Depths for Sections  %%%%%%%%%%%%%%

% 1) Section
% 1.1) SubSection
% 1.1.1) SubSubSection
% 1.1.1.1) Paragraph
% 1.1.1.1.1) Subparagraph


\setcounter{tocdepth}{5}
\setcounter{secnumdepth}{5}


 %%%%%%%%%%%%  Set Depths for Nested Lists created by \begin{enumerate}  %%%%%%%%%%%%%%


\setlistdepth{9}
\renewlist{enumerate}{enumerate}{9}
		\setlist[enumerate,1]{label=\arabic*)}
		\setlist[enumerate,2]{label=\alph*)}
		\setlist[enumerate,3]{label=(\roman*)}
		\setlist[enumerate,4]{label=(\arabic*)}
		\setlist[enumerate,5]{label=(\Alph*)}
		\setlist[enumerate,6]{label=(\Roman*)}
		\setlist[enumerate,7]{label=\arabic*}
		\setlist[enumerate,8]{label=\alph*}
		\setlist[enumerate,9]{label=\roman*}

\renewlist{itemize}{itemize}{9}
		\setlist[itemize]{label=$\cdot$}
		\setlist[itemize,1]{label=\textbullet}
		\setlist[itemize,2]{label=$\circ$}
		\setlist[itemize,3]{label=$\ast$}
		\setlist[itemize,4]{label=$\dagger$}
		\setlist[itemize,5]{label=$\triangleright$}
		\setlist[itemize,6]{label=$\bigstar$}
		\setlist[itemize,7]{label=$\blacklozenge$}
		\setlist[itemize,8]{label=$\prime$}



 %%%%%%%%%%%%  Header here  %%%%%%%%%%%%%%


\pagestyle{fancy}
\fancyhf{}
\chead{ 
\vspace{\baselineskip}
}
\cfoot{ 
\vspace{\baselineskip}
}
\renewcommand{\headrulewidth}{0pt}
\setlength{\topsep}{0pt}\setlength{\parskip}{8.04pt}
\setlength{\parindent}{0pt}

 %%%%%%%%%%%%  This sets linespacing (verticle gap between Lines) Default=1 %%%%%%%%%%%%%%


\renewcommand{\arraystretch}{1.3}

\title{PROJECT PROPOSAL}
\date{}


%%%%%%%%%%%%%%%%%%%% Document code starts here %%%%%%%%%%%%%%%%%%%%



\begin{document}

\maketitle
\par

\par

\par

\par

{\fontsize{24pt}{28.8pt}\selectfont \textbf{Problem statement:}\par}\par

We\ use many products in our day to day life, whether it be a cellphone or a bottle of water, each of them helps us in a different way or the other. Most of the products that we use are made up of non-biodegradable substances, these substances do not degrade for many years and are also difficult to dispose of.  The products that we use become the waste for oceans, it means that the amount of non-biodegradable substances that we end up in oceans. Increasing the population increase the need for such substances which in turn increases the waste in our oceans. Not only oceans suffer but also all the water bodies that end up in oceans have to suffer too. This problem is increasing day by day and we need to stop it!!. Every year millions of plastic end up in oceans which affects marine life to a great extent. These non-biodegradable substances not only affects the marine life but also affects the life above water. Every year aquatic life gets endangered due to plastics. Not just fishes we humans also intake plastic unknowingly while drinking water. Plastic waste affects thew ecosystem and the economy of any country, it’s our responsibility to overcome it. $``$WE HUMAN’S ARE NOT TEND TO LEAVE THE EARTH BUT MEND TO SOLVE IT$"$ , it means that the problems we created must be solved by us.\par

{\fontsize{24pt}{28.8pt}\selectfont \textbf{How might we?? }\par}\par

Removing\ the plastic is an important task for cleaning our water bodies, an individual himself cannot clean the water with plastic or other non-biodegradable substances it’s next to impossible.  So how can we remove plastics??, this is the time where we use technology. We can use simple bins for collecting waste from water.\par

{\fontsize{24pt}{28.8pt}\selectfont \textbf{WORKING OF SEA BIN}\par}\par

Bins or sea bins are simple bins that can automatically collect the waste from the water with the help of a pump dragging the waste to fall in the bin later collecting the waste from the water and removing the water through the filters. Sea bins can also be controlled with the help of remote control. It has a simple structure. It is easy to use and hence less expensive. Sea bins are portable that is you can use any time. It consists of filters that are used to collect plastics which are of small sizes. The Sea bin can be deployed anywhere there is the need of plastic removed from the water bodies. The Bin will stay afloat in the water bodies $\&$  removing the plastics. The basic principle behind the bin is $``$Suction$"$ , the Pump integrated with the Bin uses the power of suction and sucks all the plastic waste that is flowing or is still around the Bin{\fontsize{16pt}{19.2pt}\selectfont . \par}Multiple Bins can be deployed at a time if the frequency of plastic waste is more, thus getting the Oceans, Rivers $\&$  Water bodies free of all plastic waste hence reducing the threats posed by these plastics to both the Humans $\&$  the marine life $\&$  animals. \par


\vspace{\baselineskip}
{\fontsize{24pt}{28.8pt}\selectfont \textbf{MATERIALS USED :}\par}\par

\begin{enumerate}
	\item Plastic bin (plastic eating plastic)\par

	\item Filters or nets\par

	\item Pipes\par

	\item Motors\par

	\item Microcontroller
\end{enumerate}\par

{\fontsize{24pt}{28.8pt}\selectfont \textbf{Feasibility:}\par}\par

The bins are Feasible with almost no running cost to it, it is ecologically designed to be the most efficient option available to us. With little to no efforts being made to get rid of this plastic issue.\par

The only work to do is to empty it every once in a while $\&$  the frequency depends on the amount of waste in its surroundings.\par


\vspace{\baselineskip}
{\fontsize{24pt}{28.8pt}\selectfont \textbf{ADVANTAGES OF SEA BIN}\par}\par

For a prototype, we can use a plastic bin for collecting plastic that is $``$\textbf{\uline{PLASTIC EATING PLASTIC$"$ . }}As they are portable we can deploy many sea bins at one place for collecting waste from water. Sea bin can also collect algae and other substances present on the upper surface of the water. As the structure of the sea bin is simple we can modify the sea bin whenever and wherever we want.\par


\vspace{\baselineskip}
{\fontsize{24pt}{28.8pt}\selectfont \textbf{Future\  opportunities:}\par}\par

On a large scale, the sea bin can collect tons of plastic with the help of sensors, we can use sensors to detect the amount of plastic in a certain area and can automatically collect the plastic and other waste. Tying up with the government, recycling companies would be a great opportunity for this initiative.\par


\vspace{\baselineskip}
{\fontsize{24pt}{28.8pt}\selectfont \textbf{MOTTO}\par}\par

{\fontsize{14pt}{16.8pt}\selectfont Waste management is as much important as cleaning waste.\par}\par


\vspace{\baselineskip}
{\fontsize{14pt}{16.8pt}\selectfont $``$OUR TRASH IS OUR WEALTH AND OUR ENVIRONMENT IS EVERYONE’S WEALTH!!!$"$ \par}\par


\vspace{\baselineskip}

\vspace{\baselineskip}

\vspace{\baselineskip}
{\fontsize{24pt}{28.8pt}\selectfont  \par}\par


\vspace{\baselineskip}

\printbibliography
\end{document}